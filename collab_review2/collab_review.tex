
\RequirePackage{lineno} 
\documentclass[superscriptaddress,showpacs,amssymb,amsmath,amsfonts,linenumbers,article]{revtex4-1}

\usepackage{array,mathtools,amssymb,booktabs}
\newcolumntype{C}{>{$}c<{$}}
\AtBeginDocument{
\heavyrulewidth=.08em
\lightrulewidth=.05em
\cmidrulewidth=.03em
\belowrulesep=.65ex
\belowbottomsep=0pt
\aboverulesep=.4ex
\abovetopsep=0pt
\cmidrulesep=\doublerulesep
\cmidrulekern=.5em
\defaultaddspace=.5em
}

%\documentclass[prc,preprint,superscriptaddress,showpacs,amssymb,amsmath,amsfonts,aps]{revtex4}
%\setlength{\topmargin}{-1.0cm}
\usepackage{graphicx}
\usepackage{dcolumn}
\newcolumntype{d}[1]{D{.}{.}{-1}} 
%\usepackage{epsfig}
\usepackage{latexsym}
\usepackage{amsmath} 
\usepackage{url}
\usepackage{natbib}
\usepackage{color}
\usepackage{mathtools}
\usepackage{framed}
\usepackage{diagbox}
\usepackage{multirow}
\usepackage{hyperref}
\usepackage{float}
%\restylefloat{table}
\usepackage[margin=20mm]{geometry}
\usepackage{enumerate}


\newcolumntype{C}[1]{>{\centering\arraybackslash}m{#1}}



%\usepackage[dvips]{color}

%\newcommand{\Dfrac}[2]{\frac{\displaystyle #1}{\displaystyle #2}}
%\newcommand{\F}[1]{Figure~\ref{#1}}
\pagenumbering{arabic}
\pdfminorversion=5 
\pdfcompresslevel=9
\pdfobjcompresslevel=9

%
\begin{document}

\begin{center}
\vspace{2cm}
{\Large ANSWERS TO THE COMMENTS OBTAINED FROM THE REVIEW COMMITTEE AFTER THE COLLABORATION REVIEW ON}\\[0.7cm]  

{\bf \large
Measurements of $\gamma_{v} p \rightarrow p' \pi^{+} \pi^{-}$ cross section with the CLAS detector for 0.4 GeV$^2 < Q^2 < 1.0$ GeV$^2$ and 1.3 GeV $< W <$ 1.825 GeV \\[0.7cm]}

by G.V. Fedotov, Iu. Skorodumina, V.D. Burkert, R.W. Gothe, K. Hicks, V.I. Mokeev,\\ and CLAS Collaboration\\[1cm]
\end{center}




{\vspace{1cm} \bf \Large \underline{The committee's comments (based on the comments by D. Carman)}}\\[0.5cm]


\begin{itemize}
\item {\bf \large Page 1:}
\begin{itemize} 


\item Note: Throughout the paper you use "\$ \textbackslash gamma\_\{\textbackslash rm v\} \$". This really should   be "\$ \textbackslash gamma\_v \$".\\
{\bf We introduced text letters instead of italic for those indices that correspond to text scripts according to the comment by Whit Armstrong received at the first round of the Ad Hoc review.}

\vspace{1em}

{\color{red} Dan's is correct about the typesetting of "\$ \textbackslash gamma\_v \$". Whit Armstrong's initial comment was for sub\/ super-script words or phrases which are introduced in this paper. The subscript "v" indicating virtual should probably be italic (i.e. default math text) everywhere}\\ 
{\color{blue} We changed "\$ \textbackslash gamma\_\{\textbackslash rm v\} \$" to "\$ \textbackslash gamma\_v \$" throughout the paper. }

\end{itemize}



\item {\bf \large  Page 4:}

\begin{itemize} 


\item Line 223. Use "... As is seen in Fig. 4, ...".\\
{\bf We like our version better.}

\vspace{1em}

{\color{red} we agree with Dan}\\ 
{\color{blue} We changed ``As it is seen" to ``As is seen".\\[0.5em]}

\item Line 257. Use "... for a charged particle track, ...".\\
{\bf The TOF system is able to provide timing information for neutral particles, too. So, we prefer to keep our version.}

\vspace{1em}

{\color{red} We suggest to change
"The CLAS TOF system provided timing information for a particle track,..." $-->$ "The CLAS TOF system provided timing information,..."}\\ 
{\color{blue} We changed the sentence according to the committee's suggestion.} 

\end{itemize}

\item {\bf \large  Page 5:}

\begin{itemize} 


\item Line 302. Use "It was shown ... position was shifted from the proton mass value and this shift depended on the CLAS sector."\\
{\bf The suggestion changes the meaning of the sentence. In the Ref.[14] (in the old paper version) the general statement, which is not based on the analyzed dataset, is given. So, we keep our version.}

\vspace{1em}

{\color{red} We guess Dan's suggestion reflects the tense consistency: turned out $-->$ depended}\\ 
{\color{blue} We changed ``turned out" to ``turns out" and hope that now the tense consistency is observed. }


\item Line 306. Use "... effect depended on the ...".\\
{\bf Kept as is for the reason described above.} 

\vspace{1em}

{\color{red} We agree to keep the present here, as this may refer to a finding 		of this analysis}\\
{\color{blue} The present tense is kept.} 



\end{itemize}

 


\item {\bf \large  Page 7:}
\begin{itemize} 

\vspace{1em}

{\color{red} } 


\item Line 421. Use "... can be reconstructed using ...".\\
{\bf We prefer to keep our version.}

\vspace{1em}

{\color{red} We agree with Dan, reconstructed or recovered}\\
{\color{blue} We changed ``restored" to ``recovered".\\[0.5em]} 


\item Line 425. Use "... final state hadrons ($X$ is the undetected particle):"\\
{\bf The $X$ there is not necessarily the undetected particle, for example for 3-pion background it corresponds to more than one particle. So, we keep our version.}

\vspace{1em}

{\color{red} We agree with you although the use of "detect" instead of "register" is preferable}\\
{\color{blue} We changed ``unregistered" to ``undetected".} 


\end{itemize}


 
\item {\bf \large  Page 12:}
\begin{itemize}


\item Line 737. Use "As is seen in ...".\\
{\bf We like our version better.}.

\vspace{1em}

{\color{red} We agree with Dan}\\ 
{\color{blue} We changed ``As it is seen" to ``As is seen".}

\end{itemize} 
 


\item {\bf \large  Page 16:}
\begin{itemize}


\item Line 947. Use "... results is based on the JM model estimations of ...".\\
{\bf We prefer to keep our version.}

\vspace{1em}

{\color{red} We agree with Dan}\\
{\color{blue} We removed ``JM model based" from this sentence, since it is already mentioned in the previous paragraph.} 


\end{itemize}  


\item {\bf \large  Page 17:}
\begin{itemize}


\item Line 1024. Use "... [4], which were obtained with a ...".\\
{\bf We prefer to keep our version.}

\vspace{1em}

{\color{red} Indeed the text suggested by Dan is more fluent with respect to cutting the sentence in two.}\\
{\color{blue} The suggested change makes the sentence overloaded by the two subordinate clauses both introduced by ``which". Additionally, this change leads to a change of the subsequent sentence that we prefer to avoid.} 


\end{itemize}  


  
\end{itemize}


{\bf \Large \underline{The committee's comments (based on the comments by V. Mokeev)}}\\[0.5cm]


\begin{itemize}

\item {\color{red} Generally speaking, we want to emphasize Viktor's comment about this paper having ``the best quality ever published for Npipi electroproduction", and that the paper should be presented "without confusing statements on data interpretations". In other words,  don't spoil the excellent results with a lengthy and muddy dissection of the model. A concise illustration of the main ingredients may suffice, just to give the reader the feeling that several resonant states are important and future analysis will provide further physics insight.}\\
{\color{blue} This is exactly that we are trying to achieve. We used the model for the estimation of the resonant contribution only and, therefore, tried to give a simple and short description of (i) what resonances were used and (ii) where their electrocouplings came from. Therefore we provided just two paragraphs describing that. All other model ingredients can be found in multiple references with the detailed model description. And we mentioned that the further analysis of the data within the model will provide further physics insight.}


\item {\color{red} As for F35(1905) and F37(1950), it may be sensible to mention them since you found a non-negligible contribution, but there are two issues. The first is that you cannot make reference to [11], where they are not even discussed. The second is the question of the $Q^2$ independence (actually this is an issue also for the first 3 states above, but removing them from the text will avoid facing the question). Obviously, assuming the electrocouplings to be $Q^2$ independent is not a valid assumption as we know that such a dependence must be there to some extent. However, our understanding was that, in absence of experimental information, you assigned them a fixed value that is deemed to provide a rough indication of their contribution, given the limited $Q^2$ of the data where they are expected to contribute. Is that actually the case? Where are the values of the electrocouplings of these two states coming from? And what do you mean exactly by $Q^2$ independence? }\\
{\color{blue} You are right, the states F35(1905) and F37(1950) are not discussed in [11], since the fit there was made only up to $W = 1.71$ GeV (the first paragraph in the Section III of the paper [11]), where they give a negligible contribution. However, the paper [11] also gives the data description up to $W = 1.85$ GeV (see Fig. 5 in Ref. [11]), and these states were included into the model anyway. In the $W$ region from 1.7 to 1.8 GeV the contribution from these states grows from 2\% to 20\% (as we mentioned in our paper).

The Ref. [23] (V. I. Mokeev et al., in Proceedings, (NSTAR 2005) 1238 (2005) pp. 47-56, arXiv:hep-ph/0512164 [hep-ph].) mentions these states explicitly and also claims the extraction of their electrocouplings (see the last two paragraphs of the  Section 3 of [23]). 

For our study we took electrocouplings of these states as they were in the model for $Q^{2} = 0.65$~GeV$^{2}$. These fixed values were used for all $Q^{2}$ bins that are reported in our paper. And yes, the reason for that is the one that you mentioned, i.e. ``in absence of experimental information, we assigned them a fixed value that is deemed to provide a rough indication of their contribution, given the limited $Q^2$ of the data where they are expected to contribute".  

The Ref. [11] was changed to Ref. [23] and the corresponding sentence on page 17 was slightly modified. If you have an idea how to describe what was done better, we will appreciate to your suggestions. 
} 


\item {\color{red} In your last paper version, starting from line 1007, you mention the states P33(1600), D15(1675) and D13(1700), referring to an approach adopted in [11]. However, in [11] it seems like only D13(1700), i.e. N(1700)3/2- in the notation there (Table III of [11]), is included in that fit, while P33(1600) is not mentioned and D15(1675) is only mentioned in passing to say that its contribution was found to be negligible. 

Since you explicitly mention that the contribution of the first 3 states is negligible, we ask you to remove any mention of such states from the paper.} \\
{\color{blue} You are right, in the paper [11] only the resonance D13(1700) was used in the fit and D15(1675) is mentioned. However, the paper [23] mentions the resonances P33(1600) and D13(1700) and even claims the extraction of their couplings. So, all the three resonances mentioned above were included into the model. For our study we took electrocouplings of these states as they were in the model for $Q^{2} = 0.65$~GeV$^{2}$.

The corresponding sentence on page 17 was modified and the reference was extended from [11] to [11,23]. We mention these resonances just to give the reader a clue, what is going on in this $W$ region and what to expect from the future publication.

If the committee insists on removal the corresponding sentence, we think that we need to repeat the calculation of the resonant part in order to make the text 100\% consistent with the plot.} 


\item p.17 lines 1011-1012... {\color{red} We think the comment is relevant and ask you to remove ref. [10] which indeed refers to a much higher $Q^2$ region.}\\
{\color{blue} The reference was removed.}



\end{itemize}

{\bf \Large \underline{Additional committee's comments}}\\[0.5cm]

 
\begin{itemize}



\item {\color{red} In addition, concerning the enumerated list at line 553 in the new paper version, we suggest to simplify it by removing the part in bold characters and by replacing the numbers with bullets (in fact you are not using those numbers 1,2,3 to identify the kinematic sets later on in the paper, right ?).}\\
{\color{blue} Removing the bold characters can confuse a reader, since it is not easy to figure out what is the particle numbering just looking at the variables, especially taking into account that permutation there is not cyclic.

We refer to the kinematic sets by their numbers in the Appendix. So, we prefer to keep it.}


\item {\color{red} Finally, we suggest to use the attached format for Table I.}\\
{\color{blue} The table format was changed according to the suggestion.}
\newcolumntype{P}[1]{>{\centering\arraybackslash}p{#1}}

\begin{table*}[htp]
\centering 
\caption{\small Number of bins for each hadronic variable \label{tab:summary_bins}}
\normalsize
  \begin{tabular}{lm{4cm}P{2cm}P{2cm}P{2cm}P{2cm}}
    \toprule
    & & \multicolumn{4}{c}{W range (GeV)} \\
    \multicolumn{2}{c}{\centering Hadronic variable }  & $1.3 - 1.35$ & $1.35 - 1.4$ & $1.4-1.45$ & $>1.45$ \\
    \cmidrule(l{5pt}r{15pt}){1-2} \cmidrule(l{5pt}r{5pt}){3-6}
    M        & Invariant mass       &   8  & 10 & 12 & 12  \\
    $\theta$ & Polar angle          &   6  & 8  & 10 & 10  \\
    $\varphi$   & Azimuthal angle      &   5  & 5  & 5  & 8   \\
    $\alpha$ & Angle between planes &   5  & 6  & 8  & 8   \\
    \bottomrule
  \end{tabular}
\end{table*}




\end{itemize}

\end{document}
