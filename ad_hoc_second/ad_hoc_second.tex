
\RequirePackage{lineno} 
\documentclass[,superscriptaddress,showpacs,amssymb,amsmath,amsfonts,linenumbers,article]{revtex4-1}
%\documentclass[prc,preprint,superscriptaddress,showpacs,amssymb,amsmath,amsfonts,aps]{revtex4}
%\setlength{\topmargin}{-1.0cm}
\usepackage{graphicx}
\usepackage{dcolumn}
\newcolumntype{d}[1]{D{.}{.}{-1}} 
%\usepackage{epsfig}
\usepackage{latexsym}
\usepackage{amsmath} 
\usepackage{url}
\usepackage{natbib}

\usepackage{mathtools}
\usepackage{framed}
\usepackage{diagbox}
\usepackage{multirow}
\usepackage{hyperref}
\usepackage{float}
%\restylefloat{table}
\usepackage[margin=20mm]{geometry}
\usepackage{enumerate}



\newcolumntype{C}[1]{>{\centering\arraybackslash}m{#1}}



%\usepackage[dvips]{color}

%\newcommand{\Dfrac}[2]{\frac{\displaystyle #1}{\displaystyle #2}}
%\newcommand{\F}[1]{Figure~\ref{#1}}
\pagenumbering{arabic}
\pdfminorversion=5 
\pdfcompresslevel=9
\pdfobjcompresslevel=9

%
\begin{document}

\begin{center}
\vspace{2cm}
{\Large ANSWERS TO THE SECOND ROUND OF AD-HOC COMMITTEE COMMENTS ON}\\[0.7cm]  

{\bf \large
Measurements of $\gamma_{v} p \rightarrow p' \pi^{+} \pi^{-}$ cross section with the CLAS detector for 0.4 GeV$^2 < Q^2 < 1.0$ GeV$^2$ and 1.3 GeV $< W <$ 1.825 GeV \\[0.7cm]}

by G.V. Fedotov, Iu. Skorodumina, V.D. Burkert, R.W. Gothe, K. Hicks, V.I. Mokeev,\\ and CLAS Collaboration\\[0.5cm]
\end{center}


Ad Hoc Committee: W. Armstrong, N. Markov, M. Ripani (chair)\\[1.5cm] 


\vspace{1cm}
{\bf We would like to thank the Ad Hoc Committee members for their comments and suggestions, which allowed to reveal many shortcomings and inconsistencies of the text and gave us an opportunity to improve the paper. Below the answers are listed.}

\vspace{1cm}


Overall Comments\\[0.5cm]

We all agree that the paper has much improved. This second round of questions and comments
addresses a few remaining points about content and style.\\[0.5cm]







\vspace{1cm}
\underline{\bf Comments by Whit Armstrong}\\[1cm]

Overall Comments\\[0.5cm]

%----------------\\[0.5cm]

This version is a substantial improvement over the previous version. The authors have addressed
almost all major issues and I would recommend they pursue publication.\\[0.5cm]


That said, there is still room for improvements that go beyond typos.\\[0.5cm]



\begin{itemize}

\item The relative resonant contributions shown FIG. 20 are very interesting. One possible feature that
may be present is a turning at Q 2 = 0.725. Of course the W~1.65 data could use some higher Q 2
points to really make this something to note but because there are no associated uncertainties on
this plot it is hard to know whether or not I should make anything of these fluctuations. Therefore
it would be useful to assign some uncertainties: experimental uncertainties (since each point
matches to data in FIG. 18) and model uncertainties.

Also, visually improving this plot would be well worth it since it is a main result. It will very likely be
cited and reproduced if it looks great. Moving the legend away from the data points, or better yet,
annotating the different W bins instead of a legend, would go a long way in making this plot look
better.\\[0.5cm]
{\bf The legend in Fig. 20 was moved away from the points. The relative resonant contribution presented in Fig. 20 was obtained as a ratio of the evaluated resonant part over the TWOPEG estimation for the full cross section. Therefore no experimental uncertainties could be assigned to the graph. Regarding the model uncertainties, it was found that the estimated resonant part of the cross section depends on the assumption for the Q2 behavior of the resonance electrocouplings. Since a fit within the JM model was not performed, the uncertainty for this estimation can hardly be evaluated. The clarification sentences were added into the last and next to last paragraphs of the section V.A.

The triangle symbols which correspond to the $W \sim 1.65$~GeV do not go further in $Q^{2}$ due to the limited $Q^{2}$ vs $W$ coverage of the analyzed dataset (see Fig. 13). The TWOPEG cross section estimation (which was used to produce Fig. 20) was performed only for those $(W,Q^{2})$ points where the experimental cross section was extracted.

Actually the Fig.20 represents a prospect for the future model analysis and is needed just to indicate a sizable resonant contribution. After the complete analysis within the model is performed, the resonant contribution will be known more precise and the correct uncertainties can be assigned.}

\end{itemize}


\#\#\# Comments on figures:\\[0.5cm]

\begin{itemize}

\item Many figures have titles which provide redundant information in the caption. IMO it is best to leave
off the titles (at the top of the graphs) unless it is really necessary.\\[0.5cm]
{\bf Figs. 3,4,5,6,10 were changed according to your suggestion.}


\end{itemize}

\#\#\# Corrections:\\[0.5cm]

\begin{itemize}

\item line 65: ``benefit from the detailed Q2 binning that is..." $\rightarrow$ ``benefit from a finer Q2 binning which is
valuable for investigating the resonant structure through ..."\\
{\bf The sentence was changed to ``Beside that, the reported cross sections benefit from a narrow $Q^{2}$ binning which is valuable for investigating the resonant structure  through establishing the $Q^{2}$-evolution of the resonance electrocouplings."}


\item  FIG. 3: Sector miss-spelled in titles (which can be removed; see comment above).\\
{\bf Done.}

\item  FIG. 4: Same as FIG. 3.\\
{\bf Done.}


\item  Eq. 10: Use latex math ``\textbackslash tan" \\[0.5cm]
{\bf Done.}

\end{itemize}



\vspace{1cm}
\underline{\bf Comments by Nikolay Markov}\\[1cm]

There is a significant improvement in every aspect of the paper.\\

I have only a few comments\\[0.5cm]

%\begin{enumerate}[A.]
\begin{enumerate}

\item I would suggest including in the introduction some of the more general remarks, i.e. what kind
of physics is being studied by investigating of the nucleon resonances and their properties.\\
{\bf We added the sentences to the end of the first paragraph and to the beginning of the fifth paragraph of the introduction. These added sentences emphasize the importance of this study for investigation of nucleon structure and the strong interaction. We hope that now the text becomes more consistent and understandable.}

\item ``Experimental setup" is mostly devoted to the target. Is it really that important?\\
{\bf There are many papers that describe the CLAS detector setup and we refer to them. In our paper we would like to avoid this repetitive information and would like to emphasize the particular features of the ``e1e" experiment. Since the target was specific for this this experiment, we focused on it.}

\item Fig. 20 -- is the second from top line (triangles) partially hidden under legend?\\
{\bf No, they are not. The legend in Fig. 20 was moved away from the points to avoid confusion. The amount of points in Fig. 20 for each $W$ range is determined by the limited $Q^{2}$ vs $W$ coverage of the analyzed dataset (see Fig. 13).}

\end{enumerate}




\vspace{1cm}
\underline{\bf Comments by Marco Ripani}\\[1cm]

\begin{itemize}


\item 85 This subsequent complete analysis $\rightarrow$ The complete analysis.\\
{\bf Done.}

\item 99 Bended $\rightarrow$ Bent.\\
{\bf Done.}

\item 113 perhaps you can avoid mentioning that the beam was polarized, since the spin information was not used in the analysis.\\
{\bf Done.}

\item 147 excluded from the consideration $\rightarrow$ excluded from the analysis\\
{\bf Done.}

\item 209 ``It turned out that the CC had some inefficient zones and their map could not be reproduced
by the Monte Carlo." $\rightarrow$ in fact I think nobody ever thought that the GSIM nphel simulation would
reproduce reality (for that you would have needed a detailed simulation of the optics). So I would
simply say, ``It turned out that the CC had some inefficient zones that could not be simulated by the
Monte Carlo technique as being too dependent on the detailed optical properties and alignment of
the mirrors.."\\
{\bf The sentence was changed to ``It turned out that the CC had some inefficient zones  that could not be simulated by the Monte Carlo technique as being too dependent on specific features of the CC design.".}

\item 214 ``Thus the inefficient zones can be differentiated from the efficient zones by more pronounced
few photoelectron peak. This fact was used for their geometrical separation." This is too vague. Can
you provide some quantitative statement about how the differentiation was made ? Was it made
by considering the average nphel ? Or what else ? We should give figures here, to make the reader
understand what was done exactly.\\
{\bf The corresponded paragraph was changed. The Eq. (1) that gives the quantitative criterion for the efficient zones selection was added. Beside that the Ref. [12] which provides all other details of the procedure was added.}

\item 235 `` Since the Monte Carlo did not reproduce the photoelectron spectrum well enough, this cut
was applied only to the experimental data, and good electrons lost in this way were recovered by
the following procedure." $\rightarrow$ ``Since we had no way of reproducing the photoelectron spectrum by
means of a Monte Carlo simulation, good electrons lost by this cut in the experimental data were
recovered by the following procedure."\\
{\bf The sentence was changed to ``Since there was no way  of reproducing the photoelectron spectrum by a Monte Carlo simulation, this cut was applied only to the experimental data, and good electrons lost in this way were recovered by the following procedure.".}

\item 318 ``had been developed" $\rightarrow$ ``have been developed"\\
{\bf The tense in the corresponded paragraph was changed from Past Perfect to Past Simple.}

\item 328 ``do not lead to the substantial" $\rightarrow$ ``do not lead to a substantial"\\
{\bf ``do not lead to the substantial distortions" was changed to ``do not lead to substantial distortions".}

\item 361 ``had the ``normal" direction of the torus magnetic field" $\rightarrow$  ``used a torus magnetic field
configuration"\\
{\bf Done.}

\item  Fig. 8 for pi-. Perhaps it would be better to show at least one slice in theta (as Fig. 8b ?) showing
how the fiducial cut catches the flat region in phi. It is difficult for the reader to figure out how that
works by just looking at the 2D figure as in this case it looks like the cut at forward angles could have
been wider.\\
{\bf We did such a conservative pi- fiducial cut for purpose. It was found that the extracted differential cross sections are very sensitive to the selection of registered pi- i.e. a wider pi- fiducial cuts lead to the non-physical distortions in the single-differential cross sections and larger background in the missing mass distributions. This situation was also observed in other analyses of the same dataset (at USC group). So, taking into account a large analyzed statistics it was decided to keep such a conservative cut.}

\item 626-658 You should add the total number of 2pi events after all cuts, along with the Faraday cup
charge, etc., total as sum of all 4 topologies and total for each topology, just to give the reader an
idea about the total statistics analyzed.\\
{\bf The total number of 2pi events (about 1.2 million) was reported in the first paragraph the section IV.B., while the FC charge had already been reported after the Eq. (12). Beside that, the sentence in the second paragraph of the section III.D.3 (``Exclusivity cut") that describes the statistics distribution between the topologies, was modified according to the refined calculation of the total statistics. }

\item Fig. 15. The upper limit 40 in the graph is cut off.\\
{\bf Done.}

\item D. Efficiency evaluation. Can you calculate and report the average efficiency, averaged by the
number of accepted events in each 5D cell over all accepted cells ? Together with the above total
number of reconstructed events within cuts, this can give an idea of the global features of the
analysis.\\
{\bf The sentence ``The averaged (over all analyzed multi-dimensional cells) value of the efficiency was found to be about 11\%" was added after the Eq. (14).}

\item 730 ``contain many questionable cells with extremely small efficiency." $\rightarrow$ ``contain many cells with
extremely small efficiency."\\
{\bf The sentence was changed to ``Moreover, these horizontal stripes contain many cells with unreliable extremely small efficiency.".}

\item E-radiative corrections. You are separating soft photon emission by the hard one based on a 10 MeV emitted photon energy cut. But when you calculated radiated event ratios in formula (14), are you selecting radiated and non-radiated events based on the same missing mass cuts applied in the
analysis ? That is important as by selecting the same cut you make sure that the amount of ``soft"/``hard" radiation you are estimating in the correction of formula (14) is the same as actually present in the experimental data that are subject to the same cuts.\\
{\bf In Eq.(14) (Eq. (15) in the new version of the paper) we use generated (not reconstructed) radiated and non-radiated events. Both of them are not subject to any cuts. The reason for that is the following: When we calculate efficiency we use generated radiated events without any cuts and reconstructed radiated events with the same cuts as for the experimental data. The extracted in this way radiated cross section is free on any assumptions for the missing mass cuts and then is subject to the radiative corrections according to Eq.(14) (Eq.(15) in the new version). The corresponding clarifications are given after Eqs. (13) and (14) ((14) and (15) in the new version).}

\item  1023 ``The results, improve"  $\rightarrow$ ``The results improve" (remove the comma).\\
{\bf Done.}

\item 1048 ``(and therefore a model dependence of the obtained cross sections)"  $\rightarrow$ ``(with a very modest
model dependence of the obtained cross sections)" otherwise with the current wording it sounds
like the model dependence has to be counted among the improvements.\\
{\bf The sentence was changed to ``All available reaction topologies were combined together to minimize statistical uncertainties as well as a contribution from kinematic cells with zero acceptance, achieving in this way a very modest model dependence of the obtained cross sections".}

\end{itemize}

\end{document}
